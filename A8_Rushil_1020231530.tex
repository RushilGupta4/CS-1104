\documentclass[a4paper]{article}
\setlength{\topmargin}{-1.0in}
\setlength{\oddsidemargin}{-0.2in}
\setlength{\evensidemargin}{0in}
\setlength{\textheight}{10.5in}
\setlength{\textwidth}{6.5in}
\usepackage{enumitem}
\usepackage{listings}
\usepackage{amsmath}
\usepackage{hyperref}
\usepackage{amssymb}
\usepackage{tikz}
\usepackage{xcolor}
\definecolor{YellowOrange}{RGB}{255,174,66}

\hbadness=10000

\hypersetup{
    colorlinks=true,
    linkcolor=blue,
    filecolor=magenta,      
    urlcolor=cyan,
    pdftitle={Assignment 8},
    pdfpagemode=FullScreen,
    }
\def\endproofmark{$\Box$}
\newenvironment{proof}{\par{\bf Proof}:}{\endproofmark\smallskip}
\begin{document}
\begin{center}
{\large \bf \color{red}  Department of Computer Science} \\
{\large \bf \color{red}  Ashoka University} \\

\vspace{0.1in}

{\large \bf \color{blue}  Discrete Mathematics: CS-1104-1 \& CS-1104-2}

\vspace{0.05in}

    { \bf \color{YellowOrange} Assignment 8}
\end{center}
\medskip

{\textbf{Collaborators:} None} \hfill {\textbf{Name: Rushil Gupta} }

\bigskip
\hrule


\section{Straightforward}
    \begin{enumerate}
        \item \begin{enumerate}
            \item $(1001011010)_2 = (001)_2(001)_2(011)_2(010)_2 = (1232)_8$ \\

            \item $(27365)_8 = (2)(7)(3)(6)(5) = (010)_2(111)_2(011)_2(110)_2(101) = (010111011110101)_2$\\
            $ = (0010)_2(1110)_2(1111)_2(0101)_2 = (2EF5)_{16}$ \\
            
            \item $(5A2B79)_{16} = (0101)(1010)(0010)(1011)(0111)(1001)_2 = (010)(110)(100)(010)(101)(101)(111)(001)$
            $ = (2)(6)(4)(2)(5)(5)(7)(1) = (26425571)_8$ \\ \\ \\
        \end{enumerate}

        % 2. Compute the Bezout’s Coefficients for the following pairs of numbers. Show the forward and backward steps of the Euclid’s Algorithm. [4 + 4 + 4 = 12] (a) 143 and 91 (b) 1932 and 735 (c) 45 and 64
        % (a) 143 and 91 Solution: Using the Euclid’s Algorithm, we get the following steps: 143 = 91 · 1 + 52 91 = 52 · 1 + 39 52 = 39 · 1 + 13 39 = 13 · 3 + 0 So the GCD of 143 and 91 is 13. Now, we can find the Bezout’s Coefficients by working backwards. 13 = 52 − 39 · 1 = 52 − (91 − 52 · 1) · 1 = 52 − 91 + 52 = 2 · 52 − 91 = 2 · (143 − 91 · 1) − 91 = 2 · 143 − 3 · 91 Therefore, the Bezout’s Coefficients for 143 and 91 are 2 and -3. □ (b) 1932 and 735 Solution: Using the Euclids’s Algorithm we get the following steps: 1932 = 735 · 2 + 462 735 = 462 · 1 + 273 462 = 273 · 1 + 189 273 = 189 · 1 + 84 189 = 84 · 2 + 21 84 = 21 · 4 + 0 So the GCD of 1932 and 735 is 21. Now, we can find the Bezout’s Coefficients by working backwards. 21 = 189 − 84 · 2 = 189 − (273 − 189 · 1) · 2 = 189 − 273 · 2 + 189 · 2 = 189 · 3 − 273 · 2 = (462 − 273 · 1) · 3 − 273 · 2 = 462 · 3 − 273 · 5 = 462 · 3 − (735 − 462 · 1) · 5 = 462 · 8 − 735 · 5 = (1932 − 735 · 2) · 8 − 735 · 5 = 1932 · 8 − 735 · 21 □ (c) 45 and 64 Solution: Using the Euclid’s Algorithm we get the following steps: 64 = 45 · 1 + 19 45 = 19 · 2 + 7 19 = 7 · 2 + 5 7 = 5 · 1 + 2 5 = 2 · 2 + 1 2 = 1 · 2 + 0 2 So the GCD of 45 and 64 is 1. Now, we can find the Bezout’s Coefficients by working backwards. 1 = 5 − 2 · 2 = 5 − (7 − 5 · 1) · 2 = 5 · 3 − 7 · 2 = (19 − 7 · 2) · 3 − 7 · 2 = 19 · 3 − 7 · 8 = 19 · 3 − (45 − 19 · 2) · 8 = 19 · 19 − 45 · 8 = (64 − 45 · 1) · 19 − 45 · 8 = 64 · 19 − 45 · 27
        \item \begin{enumerate}
            \item The Euclid's Algorithm for 143 and 91 is as follows:
            \[
                \begin{split}
                    143 &= 91 \cdot 1 + 52 \\
                    91 &= 52 \cdot 1 + 39 \\
                    52 &= 39 \cdot 1 + 13 \\
                    39 &= 13 \cdot 3 + 0
                \end{split}    
            \]

            So, the GCD of 143 and 91 is 13. Now, we can find the Bezout's Coefficients by working backwards.

            \[
                \begin{split}
                    13 &= 52 - 39 \cdot 1 \\
                    &= 52 - (91 - 52 \cdot 1) \cdot 1 \\
                    &= 52 - 91 + 52 \\
                    &= 2 \cdot 52 - 91 \\
                    &= 2 \cdot (143 - 91 \cdot 1) - 91 \\
                    &= 2 \cdot 143 - 3 \cdot 91
                \end{split}
            \]

            Therefore, the Bezout's Coefficients for 143 and 91 are 2 and -3. \\ \\
            
            \newpage
            \item The Euclid's Algorithm for 1932 and 735 is as follows:
            \[
                \begin{split}
                    1932 &= 735 \cdot 2 + 462 \\
                    735 &= 462 \cdot 1 + 273 \\
                    462 &= 273 \cdot 1 + 189 \\
                    273 &= 189 \cdot 1 + 84 \\
                    189 &= 84 \cdot 2 + 21 \\
                    84 &= 21 \cdot 4 + 0
                \end{split}
            \]

            So, the GCD of 1932 and 735 is 21. Now, we can find the Bezout's Coefficients by working backwards.

            \[
                \begin{split}
                    21 &= 189 - 84 \cdot 2 \\
                    &= 189 - (273 - 189 \cdot 1) \cdot 2 \\
                    &= 189 - 273 \cdot 2 + 189 \cdot 2 \\
                    &= 189 \cdot 3 - 273 \cdot 2 \\
                    &= (462 - 273 \cdot 1) \cdot 3 - 273 \cdot 2 \\
                    &= 462 \cdot 3 - 273 \cdot 5 \\
                    &= 462 \cdot 3 - (735 - 462 \cdot 1) \cdot 5 \\
                    &= 462 \cdot 8 - 735 \cdot 5 \\
                    &= (1932 - 735 \cdot 2) \cdot 8 - 735 \cdot 5 \\
                    &= 1932 \cdot 8 - 735 \cdot 21
                \end{split}
            \]

            Therefore, the Bezout's Coefficients for 1932 and 735 are 8 and -21. \\ \\

            \item The Euclid's Algorithm for 45 and 64 is as follows:
            \[
                \begin{split}
                    64 &= 45 \cdot 1 + 19 \\
                    45 &= 19 \cdot 2 + 7 \\
                    19 &= 7 \cdot 2 + 5 \\
                    7 &= 5 \cdot 1 + 2 \\
                    5 &= 2 \cdot 2 + 1 \\
                    2 &= 1 \cdot 2 + 0
                \end{split}
            \]

            So, the GCD of 45 and 64 is 1. Now, we can find the Bezout's Coefficients by working backwards.

            \[
                \begin{split}
                    1 &= 5 - 2 \cdot 2 \\
                    &= 5 - (7 - 5 \cdot 1) \cdot 2 \\
                    &= 5 \cdot 3 - 7 \cdot 2 \\
                    &= (19 - 7 \cdot 2) \cdot 3 - 7 \cdot 2 \\
                    &= 19 \cdot 3 - 7 \cdot 8 \\
                    &= 19 \cdot 3 - (45 - 19 \cdot 2) \cdot 8 \\
                    &= 19 \cdot 19 - 45 \cdot 8 \\
                    &= (64 - 45 \cdot 1) \cdot 19 - 45 \cdot 8 \\
                    &= 64 \cdot 19 - 45 \cdot 27
                \end{split}
            \]

            Therefore, the Bezout's Coefficients for 45 and 64 are 19 and -27. \\ \\ \\
        \end{enumerate}

        \newpage
        \item \begin{enumerate}
            \item To prove the statement, we start by noting that:
            \[
                \text{lcm}(a, b, c) \cdot \text{gcd}(a, b, c) = a \cdot b \cdot c
            \]

            This follows from the general property for three numbers \(x, y, z\):
            \[
                \text{lcm}(x, y, z) \cdot \text{gcd}(x, y, z) = \text{lcm}(\text{gcd}(x, y), \text{gcd}(y, z), \text{gcd}(z, x)) \cdot \text{gcd}(x, y, z)
            \]

            And knowing:
            \[
                \text{lcm}(x, y) \cdot \text{gcd}(x, y) = x \cdot y
            \]

            Now, considering the properties of gcd and lcm:
            \begin{itemize}
                \item \(\text{lcm}(a, b) \cdot \text{gcd}(a, b) = a \cdot b\)
                \item \(\text{lcm}(b, c) \cdot \text{gcd}(b, c) = b \cdot c\)
                \item \(\text{lcm}(c, a) \cdot \text{gcd}(c, a) = c \cdot a\)
            \end{itemize}

            Substituting these into the equation, we get:
            \[
                a \cdot b \cdot c \cdot \text{lcm}(a, b, c) = \text{gcd}(a, b, c) \cdot \left( \frac{a \cdot b}{\text{gcd}(a, b)} \right) \cdot \left( \frac{b \cdot c}{\text{gcd}(b, c)} \right) \cdot \left( \frac{c \cdot a}{\text{gcd}(c, a)} \right)
            \]

            We then rearrange and simplify using properties of gcd (gcd is multiplicative under independent products), leading us to:
            \[
                a \cdot b \cdot c \cdot \text{lcm}(a, b, c) = \text{gcd}(a, b, c) \cdot \text{lcm}(a, b) \cdot \text{lcm}(b, c) \cdot \text{lcm}(c, a)
            \]

            \item To prove the statement, we start by noting that:
            \[
                \text{lcm}(a, b, c) \cdot \text{gcd}(a, b) \cdot \text{gcd}(b, c) \cdot \text{gcd}(c, a) = \text{gcd}(a, b, c) \cdot a \cdot b \cdot c
            \]

            Using the identity:
            \[
                \text{lcm}(x, y, z) \cdot \text{gcd}(x, y, z) = x \cdot y \cdot z
            \]

            We apply it directly here. By considering the multiplicative properties of gcd and the identity for lcm and gcd:
            \begin{itemize}
                \item \(\text{gcd}(a, b, c)\) is a factor of \(\text{gcd}(a, b)\), \(\text{gcd}(b, c)\), and \(\text{gcd}(c, a)\)
                \item \(\text{lcm}(a, b, c) \cdot \text{gcd}(a, b, c) = a \cdot b \cdot c\)
            \end{itemize}

            Thus, the product on the left:
            \[
                \text{lcm}(a, b, c) \cdot \text{gcd}(a, b) \cdot \text{gcd}(b, c) \cdot \text{gcd}(c, a)
            \]

            is equivalent to multiplying each \(a, b, c\) by their respective gcd terms, and these are all parts of the product \(a \cdot b \cdot c\), hence:

            \[
                \text{lcm}(a, b, c) \cdot \text{gcd}(a, b, c) = a \cdot b \cdot c
            \]
        \end{enumerate}

        \newpage
        % 4. Find all solutions x, if they exist, to the linear congruences. If solution does not exist, prove the same. [3 + 3 + 4 + 4 = 14] (a) 3x ≡ 4 (mod 7) (b) 5x ≡ −5 (mod 12) (c) 4x ≡ 12 (mod 8) (d) 4x ≡ 11 (mod 8)
        % Let's solve each of the linear congruences: ### Part (a) \(3x \equiv 4 \pmod{7}\) First, we find the multiplicative inverse of 3 modulo 7. We need an integer \(y\) such that \(3y \equiv 1 \pmod{7}\). By trial, checking multiples of 3: \[3 \times 1 \equiv 3 \pmod{7}\] \[3 \times 2 \equiv 6 \pmod{7}\] \[3 \times 3 \equiv 9 \equiv 2 \pmod{7}\] \[3 \times 4 \equiv 12 \equiv 5 \pmod{7}\] \[3 \times 5 \equiv 15 \equiv 1 \pmod{7}\] Thus, \(y = 5\) is the inverse of 3 modulo 7. Multiplying both sides of the congruence \(3x \equiv 4 \pmod{7}\) by 5 gives: \[5 \cdot 3x \equiv 5 \cdot 4 \pmod{7}\] \[x \equiv 20 \equiv 6 \pmod{7}\] **Solution**: \(x \equiv 6 \pmod{7}\) ### Part (b) \(5x \equiv -5 \pmod{12}\) We seek the multiplicative inverse of 5 modulo 12. Checking multiples of 5: \[5 \times 1 \equiv 5 \pmod{12}\] \[5 \times 2 \equiv 10 \pmod{12}\] \[5 \times 3 \equiv 15 \equiv 3 \pmod{12}\] \[5 \times 5 \equiv 25 \equiv 1 \pmod{12}\] So, the inverse is 5. Multiplying the congruence \(5x \equiv -5 \pmod{12}\) by 5: \[25x \equiv -25 \pmod{12}\] \[x \equiv -25 \equiv 11 \pmod{12}\] (since \(-25 \equiv 11 \pmod{12}\)) **Solution**: \(x \equiv 11 \pmod{12}\) ### Part (c) \(4x \equiv 12 \pmod{8}\) Simplify the congruence: \[4x \equiv 12 \equiv 4 \pmod{8}\] Dividing through by 4 (noting gcd(4,8) divides 4): \[x \equiv 1 \pmod{2}\] **Solution**: \(x \equiv 1 \pmod{2}\) (all odd numbers) ### Part (d) \(4x \equiv 11 \pmod{8}\) First, reduce \(11 \pmod{8}\): \[4x \equiv 11 \equiv 3 \pmod{8}\] The equation \(4x \equiv 3 \pmod{8}\) suggests no solution because 4x mod 8 can only be 0, 4 due to \(x\) being an integer, and \(4x\) is always even while 3 is odd. **Conclusion**: No solution exists for part (d). These are the solutions or non-solutions for each part of the problem.
        \item \begin{enumerate}
            \item To solve the linear congruence \(3x \equiv 4 \pmod{7}\), we first find the multiplicative inverse of 3 modulo 7. We need an integer \(y\) such that \(3y \equiv 1 \pmod{7}\). By trial, checking multiples of 3:
            \[
                \begin{split}
                    3 \times 1 &\equiv 3 \pmod{7} \\
                    3 \times 2 &\equiv 6 \pmod{7} \\
                    3 \times 3 &\equiv 9 \equiv 2 \pmod{7} \\
                    3 \times 4 &\equiv 12 \equiv 5 \pmod{7} \\
                    3 \times 5 &\equiv 15 \equiv 1 \pmod{7}
                \end{split}
            \]

            Thus, \(y = 5\) is the inverse of 3 modulo 7. Multiplying both sides of the congruence \(3x \equiv 4 \pmod{7}\) by 5 gives:
            \[
                \begin{split}
                    5 \cdot 3x &\equiv 5 \cdot 4 \pmod{7} \\
                    x &\equiv 20 \equiv 6 \pmod{7}
                \end{split}
            \]

            Therefore, the solution to the linear congruence is \(x \equiv 6 \pmod{7}\). \\

            \item To solve the linear congruence \(5x \equiv -5 \pmod{12}\), we seek the multiplicative inverse of 5 modulo 12. Checking multiples of 5:
            \[
                \begin{split}
                    5 \times 1 &\equiv 5 \pmod{12} \\
                    5 \times 2 &\equiv 10 \pmod{12} \\
                    5 \times 3 &\equiv 15 \equiv 3 \pmod{12} \\
                    5 \times 5 &\equiv 25 \equiv 1 \pmod{12}
                \end{split}
            \]

            So, the inverse is 5. Multiplying the congruence \(5x \equiv -5 \pmod{12}\) by 5:
            \[
                \begin{split}
                    25x &\equiv -25 \pmod{12} \\
                    x &\equiv -25 \equiv 11 \pmod{12}
                \end{split}
            \]

            Therefore, the solution to the linear congruence is \(x \equiv 11 \pmod{12}\). \\

            \item To solve the linear congruence \(4x \equiv 12 \pmod{8}\), we simplify the congruence:
            \[
                4x \equiv 12 \equiv 4 \pmod{8}
            \]

            Dividing through by 4 (noting gcd(4, 8) divides 4):
            \[
                x \equiv 1 \pmod{2}
            \]

            Therefore, the solution to the linear congruence is \(x \equiv 1 \pmod{2}\) (all odd numbers). \\

            \item To solve the linear congruence \(4x \equiv 11 \pmod{8}\), we first reduce \(11 \pmod{8}\):
            \[
                4x \equiv 11 \equiv 3 \pmod{8}
            \]

            The equation \(4x \equiv 3 \pmod{8}\) suggests no solution because \(4x \mod 8\) can only be 0, 4 due to \(x\) being an integer, and \(4x\) is always even while 3 is odd.
        \end{enumerate}


        \newpage
        % Find all solutions x, if they exist, to the system of congruences. If solution does not exist, prove the same. [3 + 3 + 2 + 2 + 4 + 6 = 20] 1 (a) x ≡ 3 (mod 5) (1) x ≡ 2 (mod 7) (2) x ≡ 5 (mod 11) (3) (b) x ≡ 5 (mod 6) (1) x ≡ 2 (mod 35) (2) x ≡ 47 (mod 143) (3) (c) 11x ≡ 33 (mod 55) (1) 5x ≡ 10 (mod 35) (2) 7x ≡ 35 (mod 77) (3) (d) x ≡ 5 (mod 6) (1) 2x ≡ 6 (mod 8) (2) (e) 2x + 3y ≡ 0 (mod 7) (1) 5x + 4y ≡ 2 (mod 11) (2) (f) x + 2y + 3z ≡ 2 (mod 3) (1) 2x + 3y + z ≡ 1 (mod 5) (2) 3x + y + 2z ≡ 4 (mod 7) (3)
        % x ≡ 3 (mod 5) x ≡ 2 (mod 7) x ≡ 5 (mod 11) Solution: We can solve the system of congruences using the Chinese Remainder Theorem. We first find the product of the moduli, N = 5 · 7 · 11 = 385. Now we can find the partial products, M1 = 385/5 = 77, M2 = 385/7 = 55, and M3 = 385/11 = 35. Now we can find the modular inverses of the partial products with respect to the moduli using the Euclid’s Algorithm, and Bezout’s Coefficients. ˆ inverse of 77 modulo 5 The inverse of 77 modulo 5 is 3, since 77 · 3 ≡ 231 ≡ 1 (mod 5). ˆ inverse of 55 modulo 7 The inverse of 55 modulo 7 is 6, since 55 · 6 ≡ 330 ≡ 1 (mod 7). ˆ inverse of 35 modulo 11 The inverse of 35 modulo 11 is 6, since 35 · 6 ≡ 210 ≡ 1 (mod 11). Now we can find the solution to the system of congruences. X = 3 · 77 · 3 + 2 · 55 · 6 + 5 · 35 · 6 = 693 + 660 + 1050 = 2403 ≡ 385 · 6 + 93 ≡ 93 (mod 385)
        \item \begin{enumerate}
            \item We have the following congruences:
            \begin{itemize}
                \item \(x \equiv 3 \pmod{5}\)
                \item \(x \equiv 2 \pmod{7}\)
                \item \(x \equiv 5 \pmod{11}\)
            \end{itemize}

            We can solve the system of congruences using the Chinese Remainder Theorem. We first find $M = 5 \cdot 7 \cdot 11 = 385$. Now we can find the partial products, $M_1 = 385/5 = 77$, $M_2 = 385/7 = 55$, and $M_3 = 385/11 = 35$. Now we can find the modular inverses of the partial products with respect to the moduli.\\

            \begin{itemize}
                \item Inverse of 77 modulo 5: The inverse of 77 modulo 5 is 3, since \(77 \cdot 3 \equiv 231 \equiv 1 \pmod{5}\).
                \item Inverse of 55 modulo 7: The inverse of 55 modulo 7 is 6, since \(55 \cdot 6 \equiv 330 \equiv 1 \pmod{7}\).
                \item Inverse of 35 modulo 11: The inverse of 35 modulo 11 is 6, since \(35 \cdot 6 \equiv 210 \equiv 1 \pmod{11}\).
            \end{itemize}

            Now we can find the solution to the system of congruences:
            \[
                \begin{split}
                    x &= 3 \cdot 77 \cdot 3 + 2 \cdot 55 \cdot 6 + 5 \cdot 35 \cdot 6 \\
                    &= 693 + 660 + 1050 \\
                    &= 2403 \equiv 385 \cdot 6 + 93 \equiv 93 \pmod{385}
                \end{split}
            \]

            \item We have the following congruences:
            \begin{itemize}
                \item \(x \equiv 5 \pmod{6}\)
                \item \(x \equiv 2 \pmod{35}\)
                \item \(x \equiv 37 \pmod{143}\)
            \end{itemize}

            We can solve the system of congruences using the Chinese Remainder Theorem. We first find $N = 6 \cdot 35 \cdot 143 = 30030$. Now we can find the partial products, $M_1 = 30030/6 = 5005$, $M_2 = 30030/35 = 858$, and $M_3 = 30030/143 = 210$. Now we can find the modular inverses of the partial products with respect to the moduli.\\

            \begin{itemize}
                \item Inverse of 5005 modulo 6: The inverse of 5005 modulo 6 is 1, since \(5005 \cdot 1 \equiv 1 \pmod{6}\).
                \item Inverse of 858 modulo 35: The inverse of 858 modulo 35 is 2, since \(858 \cdot 2 \equiv 1 \pmod{35}\).
                \item Inverse of 210 modulo 143: The inverse of 210 modulo 143 is 111, since \(210 \cdot 111 \equiv 1 \pmod{143}\).
            \end{itemize}

            Now we can find the solution to the system of congruences:
            \[
                \begin{split}
                    x &= 5 \cdot 5005 \cdot 1 + 2 \cdot 858 \cdot 2 + 37 \cdot 210 \cdot 111 \\
                    &= 25025 + 3432 + 862470 \\
                    &= 890927 \equiv 20057 \pmod{30030}
                \end{split}
            \]

            \item We have the following congruences:
            \begin{itemize}
                \item \(11x \equiv 33 \pmod{55}\)
                \item \(5x \equiv 10 \pmod{35}\)
                \item \(7x \equiv 35 \pmod{77}\)
            \end{itemize}

            First, we can divide each of the congruences to simplify them as follows:
            \begin{itemize}
                \item \(x \equiv 3 \pmod{5}\)
                \item \(x \equiv 2 \pmod{7}\)
                \item \(x \equiv 5 \pmod{11}\)
            \end{itemize}

            From part (a), we know that the solution to this system of congruences is \(x \equiv 93 \pmod{385}\).


            % (d)  x ≡ 5 (mod 6) (1) 2x ≡ 6 (mod 8) (2)
            \item We have the following congruences:
            \begin{itemize}
                \item \(x \equiv 5 \pmod{6}\)
                \item \(2x \equiv 6 \pmod{8} \equiv x \equiv 3 \pmod{4}\)
            \end{itemize}

            By brute force, we can see that $x = 11$ satisfies the equations.



                

        \end{enumerate}

    \end{enumerate}

\newpage
\section{$\neg$Straightforward}
    \begin{enumerate}
        \item a

    \end{enumerate}

\newpage
\section{Bonus}
    \begin{enumerate}
        \item a
    \end{enumerate}

\end{document}