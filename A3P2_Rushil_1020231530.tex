\documentclass[a4paper]{article}
\setlength{\topmargin}{-1.0in}
\setlength{\oddsidemargin}{-0.2in}
\setlength{\evensidemargin}{0in}
\setlength{\textheight}{10.5in}
\setlength{\textwidth}{6.5in}
\usepackage{enumitem}
\usepackage{amsmath}
\usepackage{hyperref}
\usepackage{amssymb}
\usepackage{graphicx}
\usepackage[dvipsnames] {xcolor}
\usepackage{mathpartir}

\hbadness=10000

\hypersetup{
    colorlinks=true,
    linkcolor=blue,
    filecolor=magenta,      
    urlcolor=cyan,
    pdftitle={Assignment 3 Part 2},
    pdfpagemode=FullScreen,
    }
\def\endproofmark{$\Box$}
\newenvironment{proof}{\par{\bf Proof}:}{\endproofmark\smallskip}
\begin{document}
\begin{center}
{\large \bf \color{red}  Department of Computer Science} \\
{\large \bf \color{red}  Ashoka University} \\

\vspace{0.1in}

{\large \bf \color{blue}  Discrete Mathematics: CS-1104-1 \& CS-1104-2}

\vspace{0.05in}

    { \bf \color{YellowOrange} Assignment 3 Part 2}
\end{center}
\medskip

{\textbf{Collaborators:} None} \hfill {\textbf{Name: Rushil Gupta} }

\bigskip
\hrule


% Begin your assignment here %


\section{Straightforward}
    \begin{enumerate}
    
    \item \begin{enumerate}
        \item We know that $(a, b) \in R^{-1} \leftrightarrow (b, a) \in R$.\\
        Let $S = R^{-1}$. So, $(a, b) \in S \leftrightarrow (b, a) \in R$.\\
        But also, $(a, b) \in S \leftrightarrow (b, a) \in S^{-1}$.\\
        So, $(a, b) \in S \leftrightarrow (b, a) \in S^{-1} \leftrightarrow (a, b) \in R$.\\
        Therefore, $S = R \equiv (R^{-1})^{-1} = R$.\\

        \item By definition, $(R \circ T) = \{(a, c) | \exists b \in S, (a, b) \in R \land (b, c) \in T\}$.\\
        So, $(a, c) \in (R \circ T)^{-1} \leftrightarrow (c, a) \in( R \circ T)$. (By definition of inverse)\\
        This implies that $\exists b \in S, (c, b) \in R \land (b, a) \in T$.\\
        Which is $(c, a) \in T^{-1} \circ R^{-1}$.\\
        Therefore, $(R \circ T)^{-1} = T^{-1} \circ R^{-1}$.\\

        \item \textbf{Reflexive:} For $(R^{-1} \circ R)$ to be reflexive, $(a, a) \in (R^{-1} \circ R)$.\\
        $\equiv \exists b \in S, (a, b) \in R^{-1} \land (b, a) \in R$.\\
        $\equiv \exists b \in S, (b, a) \in R \land (a, b) \in R$.\\
        Now, for all $a$, for some subset of $S$, the statement above holds true.\\
        Therefore, $(R^{-1} \circ R)$ is reflexive over some subset of $S$.\\

        \textbf{Symmetric:} For $(R^{-1} \circ R)$ to be symmetric, $(a, b) \in (R^{-1} \circ R) \rightarrow (b, a) \in (R^{-1} \circ R)$.\\
        Now, $(a, b) \in (R^{-1} \circ R) \equiv \exists c \in S ( (a, c) \in R \land (c, b) \in R^{-1} )$.\\
        $\equiv \exists c \in S ( (a, c) \in R \land (b, c) \in R )$. (By definition of inverse)\\
        $\equiv \exists c \in S ( (b, c) \in R \land (a, c) \in R )$.\\
        $\equiv \exists c \in S ( (b, c) \in R \land (c, a) \in R^{-1} )$.\\
        $\equiv (b, a) \in (R^{-1} \circ R)$.\\

    \end{enumerate}


    % Show the following are equivalence relations
    \item \begin{enumerate}
        \item To show that the relation is an equivalence relation, we need to show that it is reflexive, symmetric, and transitive.\\

        \textbf{Reflexive}\\
        For a function $p(n)$, $p(n) \in \Theta(g(n)) \rightarrow p(n) \in \Theta(g(n))$ (True because of the definition of $\Theta$ being a class of functions with the same time complexity).\\

        \textbf{Symmetric}\\
        We need to show that if $p(n) \in \Theta(g(n))$, then $q(n) \in \Theta(g(n)) \rightarrow p(n) \in \Theta(g(n))$.\\
        This is true because $\Theta$ is a class of functions with the same time complexity. So, if $p(n)$ and $q(n)$ have the same time complexity (Given by them being part of $\Theta(g(n))$), then $p(n)$ and $q(n)$ are symmetric.\\

        \textbf{Transitive}\\
        Consider some $f(n)$ such that $f(n) \in \Theta(g(n))$.\\
        Now, $p(n) \in \Theta(g(n)) \rightarrow q(n) \in \Theta(g(n)) \rightarrow f(n) \in \Theta(g(n))$.\\
        So, $p(n) \in \Theta(g(n)) \rightarrow f(n) \in \Theta(g(n))$.\\
        Therefore, the relation is transitive.\\

        Hence, the relation is an equivalence relation.\\

        \newpage

        % R2 where R is any symmetric relation over some subset of the original set.
        \item First, note that $R^2 \equiv (R \circ R) \equiv \{(a, b) | \exists c \in S, (a, c) \in R \land (c, b) \in R\}$.\\
        Now, we need to show that $R^2$ is an equivalence relation.\\

        \textbf{Reflexive}\\
        For $R^2$ to be reflexive, $(a, a) \in R^2$.\\
        $\equiv \exists c \in S, (a, c) \in R \land (c, a) \in R$.\\
        Since we know $R$ is symmetric, $(a, b) \in R \rightarrow (b, a) \in R$.\\
        So, $(a, a) \in R^2$.\\

        \textbf{Symmetric}\\
        For $R^2$ to be symmetric, $(a, b) \in R^2 \rightarrow (b, a) \in R^2$.\\
        $LHS \equiv \exists c \in S, (a, c) \in R \land (c, b) \in R$.\\

        Since $R$ is symmetric, $(a, b) \in R \rightarrow (b, a) \in R$.\\
        So, $LHS \equiv \exists c \in S, (b, c) \in R \land (c, a) \in R$.\\
        $\equiv (b, a) \in R^2$.\\

        $\therefore R^2$ is symmetric.\\

        \textbf{Transitive}\\
        For $R^2$ to be transitive, $(a, b) \in R^2 \land (b, c) \in R^2 \rightarrow (a, c) \in R^2$.\\
        $LHS \equiv (\exists d \in S ((a, d) \in R \land (d, b) \in R)) \land (\exists e \in S ((b, e) \in R \land (e, c) \in R$)).\\
        $\equiv \exists d, e \in S ((a, d) \in R \land (d, b) \in R \land (b, e) \in R \land (e, c) \in R)$.\\
        
        Setting $d = e$, we get:\\
        $\equiv \exists d \in S ((a, d) \in R \land (d, b) \in R \land (b, d) \in R \land (d, c) \in R)$.\\
        $((d, b) \in R \land (b, d) \in R) \equiv TRUE$ (Since $R$ is symmetric).\\
        So, $\exists d \in S ((a, d) \in R \land (d, c) \in R)$.\\
        $\equiv (a, c) \in R^2$.\\

        $\therefore R^2$ is transitive.\\

        Therefore, $R^2$ is an equivalence relation.\\

        \item Let $A$ and $B$ be two sets.\\
        By the CBS Theorem, since $f$ and $g$ and injective maps, $|A| \leq |B|$ and $|B| \leq |A| \rightarrow |A| = |B|$.\\

        \textbf{Reflexive}\\
        We know, $|A| = |A| \land |B| = |B|$.\\
        So, $(A, A) \in R \land (B, B) \in R$.\\

        \textbf{Symmetric}\\
        We know $|A| = |B| \rightarrow |B| = |A|$.\\
        So, $(A, B) \in R \rightarrow (B, A) \in R$.\\

        \textbf{Transitive}\\
        We know $|A| = |B| \land |B| = |C| \rightarrow |A| = |C|$.\\
        So, $(A, B) \in R \land (B, C) \in R \rightarrow (A, C) \in R$.\\

        Therefore, the relation is an equivalence relation.\\


        % R is a relation over Z, defined by, R = {(a, b) : a, b ∈ Z ∧ n|(a − b)} for some fixed n ∈ Z, n ̸= 0.
        \item Let $R = \{(a, b) | a, b \in Z \land n|(a - b)\}$ for some fixed $n \in Z, n \neq 0$.\\
        
        \textbf{Reflexive}\\
        For $R$ to be reflexive, $(a, a) \in R$.\\
        $\equiv n|(a - a) \equiv n|0$.\\
        Since $n \neq 0$, $n|0$ is true trivially.\\

        \textbf{Symmetric}\\
        For $R$ to be symmetric, $(a, b) \in R \rightarrow (b, a) \in R$.\\
        $\equiv n|(a - b) \rightarrow n|(b - a)$.\\ 

        Note that $n|a \rightarrow n|-a$ since our domain of discourse is integers.\\
        So, $LHS \equiv n|-(b - a) \equiv n|(b - a)$.\\
        Hence, $n|-(b - a) \rightarrow n|(b - a)$.\\
        Therefore, $R$ is symmetric.\\

        \textbf{Transitive}\\
        For $R$ to be transitive, $(a, b) \in R \land (b, c) \in R \rightarrow (a, c) \in R$.\\
        $\equiv n|(a - b) \land n|(b - c) \rightarrow n|(a - c)$.\\
        
        Now, looking at the L.H.S:\\
        Let $p \cdot n = a - b$ and $q \cdot n = b - c$.\\
        So, $a - c = (p + q) \cdot n$.\\
        Since $p, q \in Z$, $a - c \in Z$.\\
        So, $n|(a - c)$.\\

        Therefore, $R$ is transitive.\\

        Hence, $R$ is an equivalence relation.\\
    \end{enumerate}

    \newpage
    \item \begin{enumerate}
        \item The diagram for the given relation is as follows:
        \begin{figure}[ht]
            \centering
            \includegraphics[width=0.35\linewidth]{"DM A3P2 Diagram 1.png"}
        \end{figure} \\

        \item The diagram for the given relation is as follows:
        \begin{figure}[ht]
            \centering
            \includegraphics[width=0.6\linewidth]{"DM A3P2 Diagram 2.png"}
        \end{figure}
    \end{enumerate}

    \item \begin{enumerate}
        \item The assertion that a Hasse diagram never contains triangles, or three mutually joined elements, is valid. \\
        
        The existence of three mutually connected elements implies there exists a path from the first to the second, the second to the third, and from the third back to the first (or from the first directly to the third). This scenario is incompatible as it would indicate that the relationship does not establish a Partially Ordered Set (POSET), formally indicated by: \\
         \[(a, b) \in R \land (b, c) \in R \land (c, a) \in R\]
         
        If a path from the first to the third is implied, it is due to the transitive nature of the relation, thus not depicted in the Hasse diagram, leading to the absence of triangles. \\

        \item The statement that a Hasse diagram for a total order relation on a finite set can always be depicted on a single line is accurate. A total order is a POSET where every element is comparable to every other, allowing for a linear arrangement of elements where each is directly comparable to the others. Thus, in the Hasse diagram of a total order, elements can be connected by at most two edges, one to the subsequent element and one to the preceding, with the maximum and minimum elements having only one edge, leading to or from them, respectively.
    \end{enumerate}

    \item The matrix \(A = [a_{ij}]\) is symmetric due to the symmetric nature of \(R\). Furthermore, the given rule implies that each row of \(A\) contains at most one \(1\). Now, consider \(A^2 = B = [b_{ij}]\), where \(b_{ij} = \sum_{k=1}^{n} a_{ik} \times a_{kj}\). \\

    We observe that if \(k \neq j\) and \(k \neq i\), then \(b_{ij} = 0\). This is because, by symmetry, the product \(a_{ik}a_{kj} = a_{ki}a_{kj}\), and by our rule, if \(i \neq j\), then either \((i, k) \notin R\) or \((k, j) \notin R\). Thus, \(b_{ij}\) is only non-zero when \(i = j\), indicating that \(A^2\) is a diagonal matrix.

    \newpage
    \item \begin{enumerate}
        \item This will be proven using weak induction.
        
        \textbf{Base Case}\\
        For \(n = 1\), \(A^1 = A = \begin{bmatrix} 1 & 1 \\ 0 & 1 \end{bmatrix}\).\\
        So, the base case holds true.\\

        \textbf{Inductive Hypothesis}\\
        Assume that \(A^k = \begin{bmatrix} 1 & k \\ 0 & 1 \end{bmatrix}\) for some \(k \in \mathbb{N}\).\\

        \textbf{Inductive Step}\\
        Now, \(A^{k+1} = A^k \cdot A = \begin{bmatrix} 1 & k \\ 0 & 1 \end{bmatrix} \cdot \begin{bmatrix} 1 & 1 \\ 0 & 1 \end{bmatrix}\)\\
        \(= \begin{bmatrix} 1 \cdot 1 + k \cdot 0 & 1 \cdot 1 + k \cdot 1 \\ 0 \cdot 1 + 1 \cdot 0 & 0 \cdot 1 + 1 \cdot 1 \end{bmatrix}\)\\
        \(= \begin{bmatrix} 1 & k+1 \\ 0 & 1 \end{bmatrix}\)\\

        Therefore, by weak induction, \(A^n = \begin{bmatrix} 1 & n \\ 0 & 1 \end{bmatrix}\) for all \(n \in \mathbb{N}\).\\ \\


        % C = [[0, 1], [1, 1]]. Show that C^n is [[F_{n-1}, F_n], [F_n, F_{n+1}]]. Prove Det(C^n) = (-1)^n
        \item This will be proven using weak induction.

        \textbf{Base Case}\\
        For \(n = 1\), \(C^1 = C = \begin{bmatrix} 0 & 1 \\ 1 & 1 \end{bmatrix}\).\\
        So, the base case holds true, since \(F_0 = 0\) and \(F_1 = 1\).\\

        \textbf{Inductive Hypothesis}\\
        Assume that \(C^k = \begin{bmatrix} F_{k-1} & F_k \\ F_k & F_{k+1} \end{bmatrix}\) for some \(k \in \mathbb{N}\).\\

        \textbf{Inductive Step}\\
        Now, \(C^{k+1} = C^k \cdot C = \begin{bmatrix} F_{k-1} & F_k \\ F_k & F_{k+1} \end{bmatrix} \cdot \begin{bmatrix} 0 & 1 \\ 1 & 1 \end{bmatrix}\)\\
        \(= \begin{bmatrix} F_{k-1} \cdot 0 + F_k \cdot 1 & F_{k-1} \cdot 1 + F_k \cdot 1 \\ F_k \cdot 0 + F_{k+1} \cdot 1 & F_k \cdot 1 + F_{k+1} \cdot 1 \end{bmatrix}\)\\
        \(= \begin{bmatrix} F_k & F_{k+1} \\ F_{k+1} & F_{k+1} + F_k \end{bmatrix}\) (By the definition of Fibonacci numbers)\\
        \(= \begin{bmatrix} F_k & F_{k+1} \\ F_{k+1} & F_{k+2} \end{bmatrix}\) (By the definition of Fibonacci numbers)\\

        Therefore, by weak induction, \(C^n = \begin{bmatrix} F_{n-1} & F_n \\ F_n & F_{n+1} \end{bmatrix}\) for all \(n \in \mathbb{N}\).\\

        Now, \(\text{Det}(C^n) = (-1)^n\) needs to be proven.\\
        Note that \(\text{Det}(C^n) = \text{Det}(C) \times \text{Det}(C) \times \dots \times \text{Det}(C)\).\\
        Also, \(\text{Det}(C) = 0 \cdot 1 - 1 \cdot 1 = -1\).\\
        So, \(\text{Det}(C^n) = (-1)^n\).\\ \\ \\ \\

        \end{enumerate}

        \item \begin{enumerate}
            \item The given matrix is \(A = \begin{bmatrix} 1 & 2 & 3 & 3 \\ 2 & 1 & 1 & 1 \\ 3 & 6 & 5 & 4 \\ 3 & 3 & 2 & 2 \end{bmatrix}\).\\
            
            Now, we can use elementary row operations.
            $R_4 \leftarrow R_4 - 3R_1, R_3 \leftarrow R_3 - 3R_1, R_2 \leftarrow R_2 - 2R_1$\\
            \[A = \begin{bmatrix} 1 & 2 & 3 & 3 \\ 0 & -3 & -5 & -5 \\ 0 & 0 & -4 & -5 \\ 0 & -3 & -7 & -7 \end{bmatrix}\]

            $R_4 \leftarrow R_4 - R_2$\\
            \[A = \begin{bmatrix} 1 & 2 & 3 & 3 \\ 0 & -3 & -5 & -5 \\ 0 & 0 & -4 & -5 \\ 0 & 0 & -2 & -2 \end{bmatrix}\]

            $R_4 \leftarrow R_4 - \frac{1}{2}R_3$\\
            \[A = \begin{bmatrix} 1 & 2 & 3 & 3 \\ 0 & -3 & -5 & -5 \\ 0 & 0 & -4 & -5 \\ 0 & 0 & 0 & 0.5 \end{bmatrix}\]

            Since no scalar multiplication was done on the rows, the determinant of the matrix is the product of the diagonal elements.\\
            \[\text{Det}(A) = 1 \times -3 \times -4 \times 0.5 = 6\]

            Since determinant is non-zero, reducing the matrix to the identity matrix will give us the inverse.\\
            
            $R_4 \leftarrow 2R_4, R_3 \leftarrow -\frac{1}{4}R_3, R_2 \leftarrow -\frac{1}{3}R_2$\\
            \[A = \begin{bmatrix} 1 & 2 & 3 & 3 \\ 0 & 1 & \frac{5}{3} & \frac{5}{3} \\ 0 & 0 & 1 & \frac{5}{4} \\ 0 & 0 & 0 & 1 \end{bmatrix}\]

            $R_3 \leftarrow R_3 - \frac{5}{4}R_4, R_2 \leftarrow R_2 - \frac{5}{3}R_4, R_1 \leftarrow R_1 - 3R_4$\\
            \[A = \begin{bmatrix} 1 & 2 & 3 & 0 \\ 0 & 1 & \frac{5}{3} & 0 \\ 0 & 0 & 1 & 0 \\ 0 & 0 & 0 & 1 \end{bmatrix}\]

            $R_2 \leftarrow R_2 - \frac{5}{3}R_3, R_1 \leftarrow R_1 - 3R_3$\\
            \[A = \begin{bmatrix} 1 & 2 & 0 & 0 \\ 0 & 1 & 0 & 0 \\ 0 & 0 & 1 & 0 \\ 0 & 0 & 0 & 1 \end{bmatrix}\]

            $R_1 \leftarrow R_1 - 2R_2$\\
            \[A = \begin{bmatrix} 1 & 0 & 0 & 0 \\ 0 & 1 & 0 & 0 \\ 0 & 0 & 1 & 0 \\ 0 & 0 & 0 & 1 \end{bmatrix}\]
            
            To reduce A to the identity matrix, we performed the following operations: \\
            $R_4 \leftarrow R_3 - 3R_1, R_3 \leftarrow R_3 - 3R_1, R_2 \leftarrow R_2 - 2R_1$\\
            $R_4 \leftarrow R_4 - R_2$\\
            $R_4 \leftarrow R_4 - \frac{1}{2}R_3$\\
            $R_4 \leftarrow 2R_4, R_3 \leftarrow -\frac{1}{4}R_3, R_2 \leftarrow -\frac{1}{3}R_2$\\
            $R_3 \leftarrow R_3 - \frac{5}{4}R_4, R_2 \leftarrow R_2 - \frac{5}{3}R_4, R_1 \leftarrow R_1 - 3R_4$\\
            $R_2 \leftarrow R_2 - \frac{5}{3}R_3, R_1 \leftarrow R_1 - 3R_3$\\
            $R_1 \leftarrow R_1 - 2R_2$\\

            By performing these operations on the identity matrix, we get the inverse of A: 
            \[A^{-1} = \begin{bmatrix} -1/6 & 5/6 & 0 & -1/6 \\ -1/6 & -7/6 & 0 & 5/6 \\ -1/2 & 5/2 & 1 & -5/2 \\ 1 & -2 & -1 & 2 \end{bmatrix}\]


            % Find the determinant of the following matrices using Gaussian Elimination. If determinant is non-zero, then also find their inverses (you may combine multiple primitive row operations into one as long as it is clear),
            % B = [[1, 1, 1, 1], [1, 2, 3, 4], [1, 2, 6, 10], [1, 4, 10, 20]]
            \item The given matrix is \(B = \begin{bmatrix} 1 & 1 & 1 & 1 \\ 1 & 2 & 3 & 4 \\ 1 & 2 & 6 & 10 \\ 1 & 4 & 10 & 20 \end{bmatrix}\).\\

            Now, we can use elementary row operations. \\
            $R_2 \leftarrow R_2 - R_1, R_3 \leftarrow R_3 - R_1, R_4 \leftarrow R_4 - R_1$\\
            \[B = \begin{bmatrix} 1 & 1 & 1 & 1 \\ 0 & 1 & 2 & 3 \\ 0 & 1 & 5 & 9 \\ 0 & 3 & 9 & 19 \end{bmatrix}\]

            $R_3 \leftarrow R_3 - R_2, R_4 \leftarrow R_4 - 3R_2$\\
            \[B = \begin{bmatrix} 1 & 1 & 1 & 1 \\ 0 & 1 & 2 & 3 \\ 0 & 0 & 3 & 6 \\ 0 & 0 & 3 & 10 \end{bmatrix}\]

            $R_4 \leftarrow R_4 - R_3$\\
            \[B = \begin{bmatrix} 1 & 1 & 1 & 1 \\ 0 & 1 & 2 & 3 \\ 0 & 0 & 3 & 6 \\ 0 & 0 & 0 & 4 \end{bmatrix}\]

            Since no scalar multiplication was done on the rows, the determinant of the matrix is the product of the diagonal elements.\\
            \[\text{Det}(B) = 1 \times 1 \times 3 \times 4 = 12\]

            Since determinant is non-zero, reducing the matrix to the identity matrix will give us the inverse.\\

            $R_4 \leftarrow \frac{1}{4}R_4, R_3 \leftarrow \frac{1}{3}R_3$\\
            \[B = \begin{bmatrix} 1 & 1 & 1 & 1 \\ 0 & 1 & 2 & 3 \\ 0 & 0 & 1 & 2 \\ 0 & 0 & 0 & 1 \end{bmatrix}\]

            $R_3 \leftarrow R_3 - 2R_4, R_2 \leftarrow R_2 - 3R_4, R_1 \leftarrow R_1 - R_4$\\
            \[B = \begin{bmatrix} 1 & 1 & 1 & 0 \\ 0 & 1 & 2 & 0 \\ 0 & 0 & 1 & 0 \\ 0 & 0 & 0 & 1 \end{bmatrix}\]

            $R_2 \leftarrow R_2 - 2R_3, R_1 \leftarrow R_1 - R_3$\\
            \[B = \begin{bmatrix} 1 & 1 & 0 & 0 \\ 0 & 1 & 0 & 0 \\ 0 & 0 & 1 & 0 \\ 0 & 0 & 0 & 1 \end{bmatrix}\]

            $R_1 \leftarrow R_1 - R_2$\\
            \[B = \begin{bmatrix} 1 & 0 & 0 & 0 \\ 0 & 1 & 0 & 0 \\ 0 & 0 & 1 & 0 \\ 0 & 0 & 0 & 1 \end{bmatrix}\]

            \newpage
            To reduce B to the identity matrix, we performed the following operations: \\
            $R_2 \leftarrow R_2 - R_1, R_3 \leftarrow R_3 - R_1, R_4 \leftarrow R_4 - R_1$\\
            $R_3 \leftarrow R_3 - R_2, R_4 \leftarrow R_4 - 3R_2$\\
            $R_4 \leftarrow R_4 - R_3$\\
            $R_4 \leftarrow \frac{1}{4}R_4, R_3 \leftarrow \frac{1}{3}R_3$\\
            $R_3 \leftarrow R_3 - 2R_4, R_2 \leftarrow R_2 - 3R_4, R_1 \leftarrow R_1 - R_4$\\
            $R_2 \leftarrow R_2 - 2R_3, R_1 \leftarrow R_1 - R_3$\\
            $R_1 \leftarrow R_1 - 2R_2$\\

            By performing these operations on the identity matrix, we get the inverse of B:
            \[B^{-1} = \begin{bmatrix} 2 & -4/3 & 1/3 & 0 \\ -1/2 & 7/6 & -11/12 & 1/4 \\ -1 & 2/3 & 5/6 & -1/2 \\ 1/2 & -1/2 & -1/4 & 1/4 \end{bmatrix}\]

        \end{enumerate} 
    \end{enumerate}


\newpage
\section{$\neg$Straightforward}
    \begin{enumerate}
        % Let R be any equivalence relation defined on Z. For any n ∈ Z, define n = {y : y ∈ Z ∧ (n, y) ∈ R} Now we want to show, G n∈Z n = Z This notation represents a disjoint union, i.e., a collection of sets whose union is Z and intersection of any two sets is φ. This is also called a partition of Z. To show disjoint union, we do the following, (a) Show that, [ n∈Z n = Z 
        \item \begin{enumerate}
            \item Since $R$ is an equivalence relation, it is reflexive.\\ 
            This means that for any $n \in Z$, $(n, n) \in R \rightarrow n \in \overline{n}$\\
            So, $\bigcup_{n \in Z}{\overline{n}}$ contains at least all elements of $\mathbb{Z}$.\\

            Hence, $\bigcup_{n \in Z}{\overline{n}} = \mathbb{Z}$.\\

            % (b) Show that for any n, m ∈ Z, overline{n} ̸= overline{m}, we have, overline{n} ∩ overline{m} = φ
            \item Need to show that for any $n, m \in Z, n \neq m$, $\overline{n} \cap \overline{m} = \phi$.\\
            Assume that $\exists x \in \overline{n} \cap \overline{m} \rightarrow \exists x (x \in \overline{n} \land x \in \overline{m})$.\\
            By definition, $(n, x) \in R \land (m, x) \in R$.\\

            Since $R$ is an equivalence relation, it is symmetric, reflexive, and transitive. Therefore:\\
            $((m, x) \in R \rightarrow (x, m) \in R) \rightarrow (n, m) \in R$.\\

            But, $n \neq m$, so this is a contradiction.\\

            Therefore, $\overline{n} \cap \overline{m} = \phi$.\\
        \end{enumerate}


        \item \begin{enumerate}[label=(\roman*)]
            \item \begin{enumerate}[label=(\alph*)]
                \item TEST
            \end{enumerate}
        \end{enumerate}
    \end{enumerate}

\newpage
\section{Bonus}
    \begin{enumerate}
        \item 
    \end{enumerate}

\end{document}