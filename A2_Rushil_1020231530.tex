\documentclass[a4paper]{article}
\setlength{\topmargin}{-1.0in}
\setlength{\oddsidemargin}{-0.2in}
\setlength{\evensidemargin}{0in}
\setlength{\textheight}{10.5in}
\setlength{\textwidth}{6.5in}
\usepackage{enumitem}
\usepackage{amsmath}
\usepackage{hyperref}
\usepackage{amssymb}
\usepackage[dvipsnames] {xcolor}
\usepackage{mathpartir}

\hbadness=10000

\hypersetup{
    colorlinks=true,
    linkcolor=blue,
    filecolor=magenta,      
    urlcolor=cyan,
    pdftitle={Assignment 2},
    pdfpagemode=FullScreen,
    }
\def\endproofmark{$\Box$}
\newenvironment{proof}{\par{\bf Proof}:}{\endproofmark\smallskip}
\begin{document}
\begin{center}
{\large \bf \color{red}  Department of Computer Science} \\
{\large \bf \color{red}  Ashoka University} \\

\vspace{0.1in}

{\large \bf \color{blue}  Discrete Mathematics: CS-1104-1 \& CS-1104-2}

\vspace{0.05in}

    { \bf \color{YellowOrange} Assignment 2}
\end{center}
\medskip

{\textbf{Collaborators:} None} \hfill {\textbf{Name: Rushil Gupta} }

\bigskip
\hrule


% Begin your assignment here %


\section{Straightforward}

    \begin{enumerate}
    \item \begin{enumerate}
        \item We know any number greater than 2 always has at least 2 integer factors (1 and itself). Therefore, if we want to define a number with only 3 factors, we need to consider the square of a number. Specifically, the square of a prime (proof will continue later).
        
        \[ \text{Let } T = p^2 \text{ where } p \text{ is a prime number.} \]

        \textbf{Proof:}
        We need to show that $p$ must be a prime. Assume that $p$ is a composite number ($p = a \times b$). Using the fundamental theorem of algebra we can uniquely factorize $a$ and $b$ into prime numbers. 

        \[\therefore a = p_1 \times p_2 \times \ldots \times p_n \text{ and } b = q_1 \times q_2 \times \ldots \times q_m \text{ where } p_i, q_i \text{ are prime numbers.}\]
        \[ \therefore p^2 = (a \times b)^2 = (p_1^2 \times p_2^2 \times \ldots \times p_n^2) \times (q_1^2 \times q_2^2 \times \ldots \times q_m^2) \]

        Note that a and b will have at least 2 distinct prime factors (Since if only 1 prime factor exists, p is the prime itself). Therefore, $p^2$ will have at least 4 factors. This is a contradiction to our assumption. Therefore, $p$ must be a prime number. \\

        \item Let P be the proposition that $p$ and $q$ are twin primes. \\
        Let Q be the proposition that $pq + 1$ is a perfect square. \\

        Note: For proposition $P$, it is enough to show $|p - q| = 2$ since we are given $p$ and $q$ to be primes. \\
    
        First we will prove $P \rightarrow Q$ is true. We know $p$ and $q$ are twin primes. WLOG, assume $p > q$.
        \[ \therefore p = q + 2 \implies pq + 1 = p(p + 2) + 1 = p^2 + 2p + 1 = (p + 1)^2 \]
        Since, $p + 1$ is an integer, $pq + 1$ is a perfect square. \\

        Now, we need to show $Q \rightarrow P$ is true. We know $\exists k (pq + 1 = k^2)$.
        \[ \therefore pq = k^2 - 1 = (k + 1)(k - 1) \]
        Since $p$ and $q$ are primes, $p$ and $q$ must be $k + 1$ and $k - 1$ respectively. Therefore, $|p - q| = 2$. \\


        Therefore, $P \leftrightarrow Q$ is true. \\ \\
        \end{enumerate}
    
    \item \begin{enumerate}
        \item Let's consider the proposition $Q \equiv$ "$P(n)$ is true $\forall n \in N$ using strong induction"
        \[ Q \equiv \forall n ((P(1) \land P(2) \land \dots \land P(n)) \rightarrow P(n+1)) \]

        Now lets, consider the proposition $R \equiv$ "$P(n)$ is true $\forall n \in N$ using weak induction"
        \[ R \equiv \forall n (P(n) \rightarrow P(n+1)) \]

        We need to know if $Q \rightarrow R$ is true. \\
        
        We can show R is true if Q is true since, $Q$ has the premise $P(n)$ to conclude $P(n+1)$. Therefore, $Q \rightarrow R$ is true. \\

        \item This part will use the same proposition $Q$ and $R$ as the previous part. We need to show if $R \rightarrow Q$ is true. \\

        In $R$, we say that $P(n) \rightarrow P(n+1)$. But this comes from the idea that $P(n-1) \rightarrow P(n)$, which then comes from $P(n-2) \rightarrow P(n-1)$ and so on. This recursion only stops when we get $P(b)$ where $b$ is the base case. \\

        So, to say $R$ holds, we need to show $P(b), P(b+1), \dots, P(n)$ are true. This is exactly what $Q$ says. Therefore, $R \rightarrow Q$ is true. \\

        \item We need to show that if the Well Ordering Principle if true and we have the premises and conclusion $P(0), P(n) \rightarrow P(n+1)$, then $P(n)$ is true $\forall n \in N$.
        \begin{itemize}
            \item Let $P(n)$ be some predicate. $P(0)$ is true (as given in the premise).
            \item Start by assuming $\exists k \in N$ such that $P(k)$ is false. 
            \item Let $S = \{ n \in N | P(n) \text{ is false} \}$. 
            \item Since $S$ is a non-empty subset of $N$, it must have a smallest element $m$ (W.O.P.).
            \item Since $P(0)$ is true, $m > 0$. Therefore, $m - 1 \in N$.
            \item Since $m$ is the smallest element of $S$, $P(m - 1)$ must be true (since $m-1 \notin S$ ).
            \item Therefore, $P(m - 1) \rightarrow P(m)$ must be true (From our given premise).
            \item This is a contradiction to our assumption. Therefore, $P(n)$ is true $\forall n \in N$. \\
        \end{itemize}

        \item We need to show that if $P(n)$ is true $\forall n \in N$ using strong induction, then the Well Ordering Principle is true. \\
        \begin{itemize}
            \item Assume that the Well Ordering Principle does not hold. 
            \item This means $\exists S ((S \ne \phi) \land (S \subset N))$ such that $S$ has no smallest element.
            \item Let $P(n)$ be the proposition that $n \notin S$.
            \item We know $P(0)$ is true since $0 \notin S$ (Since if $0 \in S$ then $0$ is the least element).
            \item We need to show that $P(n) \rightarrow P(n+1)$ is true. 
            \item Assume that $P(n)$ is true. This means that $n \notin S$.
            \item Since $S$ has no smallest element, $n+1 \notin S$. Therefore, $P(n+1)$ is true.
            \item This is a contradiction to our assumption of $S \ne \phi$ since $\forall n (n \notin S)$ by strong induction.
        \end{itemize}

        Therefore, the Well Ordering Principle is true. \\ \\
    \end{enumerate}

    \item \begin{enumerate}
        \item We need to show $\forall x, y \in \mathbb{R} ((x \ne y) \rightarrow ((x > y) \oplus (x < y)))$ is a tauology. \\
        Given $x \ne y$, consider the following: \\
        
        Assume $x > y$. Because of how this assignment is defined, $x < y$ is not possible. Therefore, $(x > y) \land \neg (x < y)$. \\

        Similarly, we casn start by assuming $x < y$. Because of how this assignment is defined, $x > y$ is not possible. Therefore, $(x < y) \land \neg (x > y)$. \\

        So, $((x > y) \land \neg (x < y)) \lor ((x < y) \land \neg (x > y)) \equiv ((x > y) \oplus (x < y))$
        \[\therefore \forall x, y \in \mathbb{R} ((x \ne y) \rightarrow ((x > y) \oplus (x < y))) \text{ is a tauology.}\]
        
        \item The set of complex numbers $\mathbb{C}$ does not satisfy the trichotomy property. This is because the operators $>$ and $<$ are not defined in this set. The only way to compare 2 complex numbers if by using their norm, which is in the set of rational numbers $\mathbb{R}$. \\
    \end{enumerate}

    \newpage
    \item We know: (i) There are $n$ teams, (ii) Each team plays all of the others exactly once, (iii) Each team loses at least one game. Need to show that there exist 2 teams which have the same number of wins. \\

    \textbf{Proof:}
    Each team plays $n-1$ matches (they play each team except themselves). Moreover, since we know each team loses at least 1 match, each team can win at most $n-2$ matches. Therefore, the possible number of wins for each team is $\{0, 1, \dots, n-2\}$ ($n-1$ options). Since there are $n$ teams, by the Pigeonhole Principle, there must be at least 2 teams with same number of wins. \\ \\


    \item \begin{enumerate}
        \item First consider the definition of the GCD function for 2 numbers:\\
        \[
            gcd(a,b) = 
            \left\{
                \begin{array}{lr}
                    a & \text{if } b = 0 \\
                    gcd(b, a \mod b) & \text{if } b \ne 0
                \end{array}
                \right\}
        \]

        We need to show that $gcd(F_{n+1}, F_n) = 1, \forall n \ge 1$. Proving using induction: \\

        \textbf{Base Case:} $n = 1$. We know $F_2 = 1$ and $F_1 = 1 \rightarrow gcd(F_2, F_1) = 1$. \\

        \textbf{Induction Hypothesis:} Assume $gcd(F_{n+1}, F_n) = 1$. We need to show $gcd(F_{n+2}, F_{n+1}) = 1$. \\
        
        \textbf{Inductive Step:} 
        \[ gcd(F_{n+1}, F_{n+2}) = gcd(F_{n+1}, F_{n+1} + F_n) \text{ (By the definition of the Fibonacci sequence)} \]
        \[ = gcd(F_{n+1}, F_n) \text{ (By the definition of the GCD function)} \]
        \[ = 1 \text{ (By the Induction Hypothesis)} \]
        
        Therefore, $gcd(F_{n+1}, F_n) = 1, \forall n \ge 1$. \\

        \item We need to show $F_{n-1} \cdot F_{n+1} - F_n^2 = (-1)^n, \forall n \ge 2$. This proof will use induction. \\
        
        \textbf{Base Case:} $n = 2 \rightarrow F_{n-1} \cdot F_{n+1} - F_n^2 = F_1 \cdot F_3 - F_2^2 = 1 \cdot 2 - 1 = 1$. \\

        \textbf{Induction Hypothesis:} Assume $F_{n-1} \cdot F_{n+1} - F_n^2 = (-1)^n$. \\

        \textbf{Inductive Step:} We need to show $F_{n} \cdot F_{n+2} - F_{n+1}^2 = (-1)^{n+1}$. \\
        \[ F_{n} \cdot F_{n+2} - F_{n+1}^2 = F_{n} \cdot (F_{n+1} + F_{n}) - F_{n+1}^2 \text{ (By the definition of the Fibonacci sequence)} \]
        \[ = F_{n+1} \cdot (F_{n} - F_{n+1}) + F_{n}^2  = F_{n+1} \cdot (-F_{n-1}) + F_{n}^2 = -(F_{n+1} \cdot F_{n-1}) + F_{n}^2 = -(-1)^n \text{ (By I.H.)} \]
        \[ = (-1)^{n+1} \]

        Therefore, $F_{n-1} \cdot F_{n+1} - F_n^2 = (-1)^n, \forall n \ge 2$. \\

        \item Need to show $F_n = \frac{\alpha^n - \beta^n}{\sqrt{5}}, \forall n \ge 0$. This proof will use induction. \\
        
        \textbf{Base Cases:} \begin{itemize}
            \item $n = 0 \rightarrow F_0 = \frac{\alpha^0 - \beta^0}{\sqrt{5}} = \frac{1 - 1}{\sqrt{5}} = 0$. \\
            \item $n = 1 \rightarrow F_1 = \frac{\alpha^1 - \beta^1}{\sqrt{5}} = \frac{\alpha - \beta}{\sqrt{5}} = 1$. \\
        \end{itemize}
        

        \textbf{Induction Hypothesis:} Assume $F_n = \frac{\alpha^n - \beta^n}{\sqrt{5}}$. We need to show $F_{n+1} = \frac{\alpha^{n+1} - \beta^{n+1}}{\sqrt{5}}$. \\

        \textbf{Inductive Step:}
        \[ F_{n+1} = F_n + F_{n-1} = \frac{\alpha^n - \beta^n}{\sqrt{5}} + \frac{\alpha^{n-1} - \beta^{n-1}}{\sqrt{5}} \]
        \[ = \frac{\alpha^{n-1}(\alpha + 1) - \beta^{n-1}(\beta + 1)}{\sqrt{5}} = \frac{\alpha^{n+1} - \beta^{n+1}}{\sqrt{5}} \text{ [Since $\alpha, \beta$ are roots of $x^2 - x - 1 = 0$]} \]
        
        Therefore, $F_n = \frac{\alpha^n - \beta^n}{\sqrt{5}}, \forall n \ge 0$. \\

    \end{enumerate}

    \newpage
    \item \begin{enumerate}
        \item For  any arbitrary $S' \subseteq S$, we need to find an $k$ such that $|S'| = k$ and $S'$ is gaurenteed to have an even number. \\
        
        Since $S = \{1, 2, \dots, 2n\}$, there are $n$ odd numbers and $n$ even numbers. Therefore, if $|S'| \ge n+1$ (or $S'$ has at least $n + 1$ elements), then $S'$ is gaurenteed to have an even number. \\

        \item Given $|S'| = n + 1$, we need to show $\exists x,y (gcd(x,y) = 1)$ is true. \\

        Since there are $2n$ possible elements, there are $n$ elements such that all the elements are not consecutive. Since $|S'| = n + 1$, there must be at least 1 pair of consecutive numbers by the Pigeonhole Principle ($n$ holes and $n+1$ pigeons). \\
        
        Let's assume these numbers are $a$ and $a+1$. Since $a$ and $a+1$ are consequetive, they are coprime. Therefore, $\exists x,y (gcd(x,y) = 1)$ is true. \\

    \end{enumerate}
    \end{enumerate}


\newpage
\section{$\neg$Straightforward}
    \begin{enumerate}
        \item Let the lightest member of the blue team is $b_1$, the second lightest member of the blue team is $b_2$ and so on. Similarly, the lightest member of the red team is $r_1$, the second lightest member of the red team is $r_2$ and so on.\\
        
        Need to show that $b_i$ can be paired with $r_i$ for all $i$, or, $\forall i \in \{1, 2, \dots, n\} (|b_i - r_i| < 1)$. \\

        \textbf{Proof using Induction:}
        \begin{itemize}
            \item \textbf{Base Case:} $i = 1$. We know $b_1$ and $r_1$ are the lightest members of their respective teams. Therefore, $b_1$ and $r_1$ can paired together (Since we are given that there is at least one valid way to pair each blue boxer with each red boxer).
            \item \textbf{Induction Hypothesis:} Assume for some $k$ $\forall i \in \{1, \cdots k\} (|b_i - r_i| < 1)$. 
            \item \textbf{Inductive Step:} We need to show $|b_{k+1} - r_{k+1}| < 1$.\\
            
            If $|b_{k+1} - r_{k+1}| < 1$ because of the weights of the boxers, then we are done. \\

            Othwerwise, we need to prove that when $|b_{k+1} - r_{k+1}| \ge 1$, there is a contradiction. \\
            If there exists a way for all boxers to be paired, then, for all $b_i$, there is a unique $r_i$ such that $|b_i - r_i| < 1$. Since $|b_{k+1} - r_{k+1}| \ge 1$, we need to check for pairing $b_{k+1}$ with $r_i$ some other $i$. \\

            \textbf{Case 1:} $b_{k+1} \ge r_{k+1} + 1$ \\
            In this case, $b_{k+1}$ cannot be paired with any $r_i$ since $r_{k+1} > r_i$ forall $i$, so $|b_{k+1} - r_{k+1}| \le |b_{k+1} - r_{k}| \le \cdots \le |b_{k+1} - r_{1}|$. This is a contradiction to our assumption that there is a valid way to pair each blue boxer with each red boxer. \\

            \textbf{Case 2:} $r_{k+1} \ge b_{k+1} - 1$ \\
            Like the arguement in Case 1, $r_{k+1}$ cannot be paired with any $b_i$ since $b_{k+1} > b_i$ forall $i$. contradiction. \\

            $\therefore |b_{k+1} - r_{k+1}| < 1$ must hold if we know that there is at least one valid way to pair each blue boxer with each red boxer. \\ \\

        \end{itemize}

        \item To show that no matter how much the bishop moves, it will always remain on a black cell, we need to show that the taxicab distance of the bishop is always even.
        
        Let the initial position of the bishop be $(x_0, y_0)$ and the final position be $(x_0 + x, y_0 + y)$. Showing that the taxi cab distance is always even using structural induction: \\
        
        \begin{itemize}
            \item \textbf{Base Case:} The bishop is at the initial position $(x_0, y_0)$. The taxicab distance is 0, which is even. \\
            \item \textbf{Induction Hypothesis:} Assume that the bishop moves some $k \ge 0$, and that the taxicab distance is even. \\
            \item \textbf{Inductive Step:} We need to show that the taxicab distance is even after the bishop moves $k+1$. \\
            The bishop can move in 4 directions: $(1, 1), (1, -1), (-1, 1), (-1, -1)$. \\
            \textbf{Case 1:} The bishop moves in the direction $(1, 1)$. \\
            The taxicab distance is $|x_0 + k + 1 - (x_0 + k)| + |y_0 + k + 1 - (y_0 + k)| = 2$, which is even. \\
            \textbf{Case 2:} The bishop moves in the direction $(1, -1)$. \\
            The taxicab distance is $|x_0 + k + 1 - (x_0 + k)| + |y_0 + k - 1 - (y_0 + k)| = 2$, which is even. \\
            \textbf{Case 3:} The bishop moves in the direction $(-1, 1)$. \\
            The taxicab distance is $|x_0 + k - 1 - (x_0 + k)| + |y_0 + k + 1 - (y_0 + k)| = 2$, which is even. \\
            \textbf{Case 4:} The bishop moves in the direction $(-1, -1)$. \\
            The taxicab distance is $|x_0 + k - 1 - (x_0 + k)| + |y_0 + k - 1 - (y_0 + k)| = 2$, which is even. \\
        \end{itemize}

        Therefore, the taxicab distance is always even. \\

        \newpage
        \item The TA's proof goes wrong in the Induction Step. Their claim is as follows:

        \begin{quote}
        We know that n students have the same major, so now consider a group of n + 1 students. Remove 1 student from this group and consider the remaining n students. These n students have the same major by Induction Hypothesis.
        \end{quote}

        This is incorrect because the Induction Hypothesis assumes that there are $n$ students in the group with the same major. Then, in TA's Induction Step, they consider $n+1$ students, but arbitrarily remove 1 student. This means that they are not actaully considering the student's major in the Induction Step. \\
        
        If the student removed is not the same student who was added in the step, the TA generalizes the Induction Hypothesis by saying there are $n$ students. This is incorrect since the student removed could have a different major. \\
    \end{enumerate}

\newpage
\section{Bonus}
    \begin{enumerate}
        \item Before I start with the sub parts, I want to add a third statement, which extends from the second statement.
        \begin{quote}
            (iii) $A \cap B = \phi = \{\}$
        \end{quote}

        \begin{enumerate}
            \item Need to show $(\forall a_1, a_2 (m \in A)) \land (\forall b_1, b_2 (n \in B))$ \\
            Since $m$ is the average of $a_1$ and $a_2$, $m$ must be between $a_1$ and $a_2$. Therefore, $m \in A$. \\
            Similarly, $n \in B$. \\

            \item Showing $P_n \in A, \forall n$ using induction: \\
            \textbf{Base Case:} $n = 0$. We know $P_0 \in A$ since $P_0 = m$ and $m \in A$ from part (a). \\
            \textbf{Induction Hypothesis:} Assume $P_k \in A$. \\
            \textbf{Inductive Step:} We need to show $P_{k+1} \in A$. \\

            We know $P_{k+1} = \frac{a_1 + P_k}{2}$. Since $P_k \in A$m $a_1 \in A$, and $P_{k+1}$ lies between $a_1$ and $P_k$, $P_{k+1} \in A$. \\
            Therefore, $P_n \in A, \forall n$. \\

            Similarly, we can show $Q_n \in A, \forall n$ (Since $Q_0 \in A$ and $Q_{k+1} = \frac{Q_k + a_2}{2}$). \\


            \item Let $p = lub(A)$ and $q = glb(B)$. \\
            Note: $\forall a \in A (a < q)$ and $\forall b \in B (b > p)$. \\
            We need to show $p = q$.
            \begin{itemize}
                \item Assume $p < q$.
                \item Since $p < q$, $\exists x (p < x < q)$.
                \item Since $x < q$, $x \in B$.
                \item Since $p < x$, $x \in A$.
                \item This is a contradiction to the fact that $A \cap B = \phi$. \\
            \end{itemize}

            Similarly, we can show $p > q$ is a contradiction. \\
            Therefore, $p = q$. \\

            \item \textbf{Base Case:} $j = i$. We know $|s - x_i| < \epsilon$ is true as its given. \\
            \textbf{Induction Hypothesis:} Assume $|s - x_j| < \epsilon$ is true for some $j$. \\
            \textbf{Inductive Step:} We need to show $|s - x_{j+1}| < \epsilon$. \\

            We know:
            \[x_j < x_{j + 1} \implies -x_j > -x_{j+1} \implies s - x_j < s - x_{j + 1}\]

            Since $s > x_i, \forall i$:
            \[s - x_j < s - x_{j + 1} \implies |s - x_j| < |s - x_{j + 1}| \]
            \[\therefore |s - x_{j + 1}| > |s - x_{j}| > \epsilon\]
            \[\therefore |s - x_{j + 1}| < \epsilon \]

            Considering $\exists k, x_k = s$: \\
            We know we can find some $x_i$ such that $|s - x_i| < \epsilon$ for any $\epsilon > 0$. \\
            But, when $k=0$, we get the limit of $x_n = s$ as $n \rightarrow \infty$. \\
            Therefore, no such $k$ exists. \\
        \end{enumerate}
    \end{enumerate}
\end{document}